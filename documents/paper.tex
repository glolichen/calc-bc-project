\documentclass{article}

\usepackage[letterpaper, margin=0.6in, bottom=1in]{geometry}
\usepackage{amsmath, amsfonts, amsthm, amssymb}
\usepackage{setspace}
\usepackage{enumitem}
\usepackage{mdframed}
\usepackage[colorlinks]{hyperref}
\usepackage{graphicx}
\usepackage{calc}
\usepackage{mlmodern}

\title{\vspace*{-40pt} Expository Materials on Solids and Surfaces of Revolution}
\author{Jayden Li}
\date{AP Calculus BC 2024-25}

\begin{document}
\setstretch{1.25}
\fontsize{11pt}{12pt}\selectfont
\setlength{\abovedisplayskip}{\abovedisplayskip/2}
\setlength{\belowdisplayskip}{\belowdisplayskip/2}
\setlength{\parindent}{0pt}
\setlength{\parskip}{2ex plus 0.5ex minus 0.2ex}
\maketitle

\begin{abstract}
	We create a series of animations using the Manim library to visualize and derive the formulas for the area of a surface of revolution and the volume of a solid of revolution, specifically disc integration and cylindrical shells. We then create a guided problem set showing students how to apply said formulas to a real-world problem.
\end{abstract}

\section{Introduction}

As an AP Calculus BC student, one of the most difficult units was on solids of revolution. While not as complicated as some other topics, it was definitely hard to visualize. Our graphing calculators (that we use, anyway), whiteboards and papers all exist in $\mathbb R^2$, while revolutions require us to visualize objects in three dimensions.

I drew heavy inspiration from Grant Sanderson's \href{https://www.youtube.com/c/3blue1brown}{3blue1brown} YouTube channel. I used his style and animation library to create these animations, and while there was learning involved, making videos with these tools was a lot easier than expected.

I hope to step through the derivation, which are included in the notes of problem sets, through visual demonstration. This will include rotating around the object and showing what a Riemann sum looks like in three dimensions.

All Manim animations are written as Python scripts. The source code for all animations can be found \href{https://github.com/glolichen/calc-bc-project}{my GitHub repository} Final videos are available on my YouTube channel.
\begin{itemize}[topsep=0pt]
	\item Disc method: \url{https://www.youtube.com/watch?v=yVkQnoRSHuQ}
	\item Cylindrical shells: TODO!
	\item Area of a surface of revoution: \url{https://www.youtube.com/watch?v=AMNdfO6yOB4}
\end{itemize}

\section{Acknowledgements}

I would like to thank the creators and maintainers of \href{https://www.manim.community/}{Manim Community Edition} and its documentation. Clearly, my project is impossible without their dedication to the Manim library.

It is clear my project has significant stylistic similarities to 3blue1brown, whose videos have inspired be significantly and made me see the power and beauty of visualization and visual derivations.

\newpage

\end{document}


